\documentclass[answers]{exam}

\usepackage{amsmath}
\usepackage{amssymb}
\usepackage{geometry}
\usepackage{venndiagram}
\usepackage{graphics}
\usepackage{graphicx}
\usepackage{tikz}

% Header and footer.
\pagestyle{headandfoot}
\runningheadrule
\runningfootrule
\runningheader{MATH 402 Applied Stochastic Processes}{Assignment 2}{Fall 2023}
\runningfooter{}{Page \thepage\ of \numpages}{}
\firstpageheader{}{}{}

\boxedpoints
\printanswers

\newcommand{\uvec}[1]{\boldsymbol{\hat{\textbf{#1}}}}
\newcommand\union\cup
\newcommand\inter\cap
\newcommand\ul\underline
\newcommand\ol\overline

\title{Assignment 2\\ MATH 402 Applied Stochastic Processes\\ Habib University -- Fall 2023}
\author{Ali Asghar Yousuf \\ ay06993}  % replace with your ID, e.g. oy02945
\date{\today}

\begin{document}
\maketitle

\begin{questions}
    \question \textbf{9.34} A server handles queries that arrive according to a Poisson process with a rate of 10 queries
    per minute.What is the probability that no queries go unanswered if the server is unavailable
    for 20 seconds?
    \begin{solution}
        \begin{align*}
            \lambda  & = 10 \text{ queries per minute}                                                             \\
            \mu      & = 60 \text{ seconds per minute}                                                             \\
            \lambda' & = \dfrac{\lambda}{\mu} = \dfrac{10}{60} = \dfrac{1}{6} \text{ queries per second}           \\
            t        & = 20 \text{ seconds}                                                                        \\
            P_0(t)   & = e^{-\lambda' t} = e^{-\dfrac{1}{6} \times 20} = e^{-\dfrac{10}{3}} = 0.049787068367863944
        \end{align*}
    \end{solution}

    \question \textbf{9.35} Customers deposit \$1 in a vending machine according to a Poisson process with rate $\lambda$.
    The machine issues an item with probability $p$. Find the PMF for the number of items dispensed in time $t$.

    \begin{solution}
        \begin{align*}
            \lambda  & = \text{ rate of deposit}                       \\
            p        & = \text{ probability of issuing an item}        \\
            t        & = \text{ time}                                  \\
            X        & = \text{ number of items dispensed in time } t  \\
            \text{The PMF for } X \text{ is given by:}                 \\
            P(X = k) & = \dfrac{\lambda^k e^{-\lambda}}{k!} \times p^k \\
        \end{align*}
        For example, if $p = 0.5$ and $\lambda = 2$, then the probability of dispensing $k$ items in time $t$ is given by:
        \begin{align*}
            P(X = 0) & = \dfrac{2^0 e^{-2}}{0!} \times 0.5^0 = 0.1353352832366127 \\
            P(X = 1) & = \dfrac{2^1 e^{-2}}{1!} \times 0.5^1 = 0.2706705664732254 \\
            P(X = 2) & = \dfrac{2^2 e^{-2}}{2!} \times 0.5^2 = 0.2706705664732254 \\
            \vdots
        \end{align*}
    \end{solution}

    \question \textbf{6.8} During rush hour, from 8 AM to 9 AM, traffic accidents occur according to a Poisson process
    with a rate of 5 accidents per hour. Between 9 AM and 11 AM, they occur as an independent
    Poisson process with a rate of 3 accidents per hour. What is the PMF of the total number
    of accidents between 8 AM and 11 AM?

    \begin{solution}
        Between 8 AM and 9 AM, the rate of accidents is 5 accidents per hour. Between 9 AM and 11 AM, the rate of
        accidents is 3 accidents per hour. Therefore,
        \begin{align*}
            X_1       & = \text{ number of accidents between 8 AM and 9 AM}   \\
            X_2       & = \text{ number of accidents between 9 AM and 10 AM}  \\
            X_3       & = \text{ number of accidents between 10 AM and 11 AM} \\
            \lambda_1 & = 5 \text{ accidents per hour}                        \\
            \lambda_2 & = 3 \text{ accidents per hour}                        \\
            \lambda_3 & = 3 \text{ accidents per hour}                        \\
        \end{align*}
        The rate of accidents between 8 AM and 11 AM is the sum of the rates of accidents between 8 AM and 9 AM, between 9 AM and 10 AM, and
        between 10 AM and 11 AM. Therefore,
        \begin{align*}
            \lambda & = \lambda_1 + \lambda_2 \lambda_3 = 5 + 3 + 3 = 11 \text{ accidents per hour} \\
            t       & = 3 \text{ hours}                                              \\
            X       & = \text{ number of accidents in } t \text{ hours}              \\
            \text{The PMF for } X \text{ is given by:}                               \\
            P(X = k) = \begin{cases}
                           \dfrac{e^{-11} 11^k}{k!} & \text{if } k \in \mathbb{N} \\
                           0                      & \text{otherwise}
                       \end{cases}
        \end{align*}
    \end{solution}

\end{questions}

\end{document}